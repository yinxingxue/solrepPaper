\documentclass[sigconf]{acmart}

\usepackage{amssymb}
\usepackage{paralist}
\usepackage{makecell}
\usepackage{graphicx}
%\usepackage{algorithm}
%\usepackage{algorithmic}
\usepackage{color}
\usepackage{url}
\usepackage{enumitem}
\usepackage{amsmath}

\usepackage[utf8]{inputenc}
\usepackage[T1]{fontenc}
\usepackage{microtype}
%\usepackage[pdftex,pagebackref,colorlinks]{hyperref}
\definecolor{darkblue}{rgb}{0.0,0.0,0.6}
\definecolor{darkgreen}{rgb}{0, 0.5, 0}
\definecolor{darkpurple}{rgb}{0.7, 0, 0.7}
\definecolor{red}{rgb}{1,0,0}



 \setlength{\textfloatsep}{1pt}
%\usepackage[
%		colorlinks=true,
%	\ifdefined \WithComments
%		pagebackref=true,
%	\fi
%	 citecolor=darkgreen,
%		linkcolor=darkblue,
%		urlcolor=darkpurple,
%	]{hyperref}

\usepackage{epstopdf}
\usepackage[ruled,vlined,linesnumbered]{algorithm2e}
%\usepackage[colorlinks=true,citecolor=blue,linkcolor=blue]{hyperref}
\usepackage{balance}
\usepackage{multirow}
\usepackage{bigstrut}
\usepackage{amssymb}
\usepackage{amsthm}
%\usepackage[colorlinks,linkcolor=blue,citecolor=red]{hyperref}

\newcommand{\naiveSol}{$\epsilon$-constraint}
\newcommand{\ourSol}{\textsc{SolRep}}

\newcommand{\JCS}{\mathit{JCS}}
\newcommand{\FeaSet}{\mathit{Fea(M)}}
\newcommand{\intset}{\mathcal {P} (\mathbb{Z})}
\newcommand{\sizes}[1]{\vert#1\vert}
\newcommand{\paratitle}[1]{\noindent\textbf{#1}. }
\newcommand\assign{\leftarrow}
\newcommand{\mfigure}[1]{Figure~\ref{#1}}
\newcommand{\mtable}[1]{Table~\ref{#1}}
\newcommand{\msection}[1]{Section~\ref{#1}}
\newcommand{\mequation}[1]{Equation~(\ref{#1})}
\newcommand{\malgo}[1]{Algorithm~\ref{#1}}
\newcommand{\grandr}{{\mathbb R}}
\newcommand{\grandZ}{{\mathbb Z}}
\newcommand{\grandB}{{\mathbb B}}
\newcommand{\FV}{{f_V}}
\newcommand{\Py}{\Pi}
\newcommand{\grandZPlus}{{\mathbb Z}_{> 0}}
\newtheorem{myDef}{Definition}
\newtheorem{thm}{Theorem}[section]
\newtheorem{lem}[thm]{Lemma}
\usepackage[figuresright]{rotating}

\ifdefined \WithComments
\newcommand{\xyx}[1]{\textcolor{red}{\textbf{#1}}}
\newcommand{\rev}[1]{\textcolor{red}{#1}}
\newcommand{\lyf}[1]{\textcolor{blue}{\textbf{#1}}}
\else
\newcommand{\xyx}[1]{}
\newcommand{\rev}[1]{\textcolor{black}{#1}}
\newcommand{\lyf}[1]{}
\fi

%\renewcommand\normalsize{%
%   \@setfontsize\normalsize\@xpt\@xiipt
%   \abovedisplayskip 1\p@ \@plus2\p@ \@minus5\p@
%   \abovedisplayshortskip \z@ \@plus3\p@
%   \belowdisplayshortskip 6\p@ \@plus3\p@ \@minus3\p@
%   \belowdisplayskip \abovedisplayskip
%   \let\@listi\@listI}

\usepackage{booktabs} % For formal tables
% Copyright
%\setcopyright{none}
%\setcopyright{acmcopyright}
%\setcopyright{acmlicensed}
\setcopyright{rightsretained}
%\setcopyright{usgov}
%\setcopyright{usgovmixed}
%\setcopyright{cagov}
%\setcopyright{cagovmixed}


% DOI
\acmDOI{10.475/123_4}

% ISBN
\acmISBN{123-4567-24-567/08/06}

%Conference
\acmConference[ICSE'18]{ACM International Conference on Software Engineering}{May 2018}{Gothenburg, Sweden}
\acmYear{2018}
\copyrightyear{2018}

\acmPrice{15.00}


\begin{document}
\title{Using Multi-Objective Integer Programming for the Optimal Feature Selection Problem}
%\titlenote{Produces the permission block, and   copyright information}
\subtitle{A Possible New Perspective on Multi-Objective Optimization Problems in SBSE}
%\subtitlenote{The full version of the author's guide is available as \texttt{acmart.pdf} document}


%\author{Yinxing Xue}
%\authornote{Dr.~Trovato insisted his name be first.}
%\orcid{1234-5678-9012}
%\affiliation{%
%  \institution{Microsoft}
%  \streetaddress{P.O. Box 1212}
%  \city{Suzhou}
%  \state{China}
%  \postcode{215002}
%}
%\email{YinXing.Xue@microsoft.com}
%
%\author{Yan-Fu Li}
%%\authornote{Dr.~Trovato insisted his name be first.}
%\orcid{1234-5678-9012}
%\affiliation{%
%  \institution{IE department, Tsinghua University}
%  \streetaddress{P.O. Box 1212}
%  \city{Beijing}
%  \state{China}
%  \postcode{100087}
%}
%\email{trovato@corporation.com}
% The default list of authors is too long for headers}
\renewcommand{\shortauthors}{Anonymous  et al.}


\begin{abstract}
%Multi-objective Evolutionary Algorithms (MOEAs) have been successfully and widely applied for solving search-based software engineering problems. For example,
The optimal feature selection problem in software product line is typically addressed by the approaches based on Indicator-based Evolutionary Algorithm (IBEA). In this study, we first expose the mathematical nature of this problem --- multi-objective binary integer linear programming. Then, we implement/propose three mathematical programming approaches to solve this problem at different scales. For small-scale problems (less than 100 features), we implement two established approaches to find all exact solutions. For medium-to-large problems (more than 100 features), we propose one efficient approach that can generate a representation of the entire Pareto front in linear time complexity. The empirical results show that our proposed method can find significantly more non-dominant solutions in similar or less execution time, in comparison with the state-of-the-art tool that combines IBEA and Differential Evolution.%\footnote{This is an abstract footnote}
\end{abstract}

%
% The code below should be generated by the tool at
% http://dl.acm.org/ccs.cfm
% Please copy and paste the code instead of the example below.
%
\begin{CCSXML}
<ccs2012>
 <concept>
  <concept_id>10010520.10010553.10010562</concept_id>
  <concept_desc>Computer systems organization~Embedded systems</concept_desc>
  <concept_significance>500</concept_significance>
 </concept>
 <concept>
  <concept_id>10010520.10010575.10010755</concept_id>
  <concept_desc>Computer systems organization~Redundancy</concept_desc>
  <concept_significance>300</concept_significance>
 </concept>
 <concept>
  <concept_id>10010520.10010553.10010554</concept_id>
  <concept_desc>Computer systems organization~Robotics</concept_desc>
  <concept_significance>100</concept_significance>
 </concept>
 <concept>
  <concept_id>10003033.10003083.10003095</concept_id>
  <concept_desc>Networks~Network reliability</concept_desc>
  <concept_significance>100</concept_significance>
 </concept>
</ccs2012>
\end{CCSXML}

\ccsdesc[500]{Computer systems organization~Embedded systems}
\ccsdesc[300]{Computer systems organization~Redundancy}
\ccsdesc{Computer systems organization~Robotics}
\ccsdesc[100]{Networks~Network reliability}


\keywords{Optimal Feature Selection Problem, Multi-Objective Optimization (MOO), Multi-Objective Integer Programming (MOIP), Indicator-Based Evolutionary Algorithm (IBEA), IBED }


\maketitle
\input{section/intro}

\input{section/problem}
\section{Mathematical Formulation of the Problem}\label{sec:formulate}
%Software companies that maintain an SPL want to respond to the various market and customers rapidly, while still considering a large amount of features and constraints among the features.
To apply IP methods, logical formulae shown in Table \ref{table:constrains} are converted into inequalities to serve as linear constraints in IP.  %An internationally accepted and widely applied standard

\subsection{Theory of Converting Logical Formulae to Inequalities}\label{sec:convert}
Converting logical formulae (LF) into inequalities is a typical problem of OR. By BIP upon the inequalities, it can be rapidly decided whether the original given LFs can be satisfied and how to be satisfied if so.
In \cite{Hooker:1988}, Hooker proposed and proved that the satisfiability problem of a conjunctive normal form (CNF) can be directly reduced to a BIP problem.
Though detailed steps are not formalized in \cite{Hooker:1988}, the basic idea of conversion was explained. %First, any arbitrary logical formula should be represented as a conjunction of \emph{1-clause}, namely \emph{clause of degree one}. Here, \emph{clause of degree} $\beta$ asserts that there are at least $\beta$  literals (corresponding to variables in BIP) to be true.
%Second,
Generally, two ways can convert an arbitrary LF as a CNF. Except for each atomic proposition, if no extra variables are introduced for operations on several atomic propositions, the running time and length of the resulting CNF can increase exponentially with the number of atomic propositions in the original formula in the worse case \cite{DBLP:journals/cor/BlairJL86}. %But if with extra variables, the conversion can finish in linear time .
%all the formulae in a system constitute the knowledge base, which is the conjunction of its formulae. For instance, constraints (1) to (13) in Table \ref{table:constrains} constitute the knowledge base of the model \emph{JCS}.

\subsection{Fast Converting Logical Formulae from the SPL Models}\label{sec:convert}
As different types of TCs and CTCs are actually not arbitrary LFs, the conversion can be done in linear time without introducing intermediate CNFs and extra variables. Let $|f_i|$ denote whether the $i$-th feature ($f_i$) is selected in a solution (see \S\ref{sec:background:mop}). We can deduce the following lemmas:
%Owing to the specific types of constraints in SPL models, the constraints are actually not arbitrary logic formulae. In this case, can the conversion be done in linear time even without introducing extra variables? The answer is yes. It is unnecessary to represent all TCs and CTCs as CNFs.

 %Different types of TCs and CTCs can be readily converted into inequalities  without introducing intermediate CNFs and extra variables. $|f_i|$ --- whether the $i$-th feature is inside the set of selected features  $F$  from $Fea(JCS)$ (see \S\ref{sec:background:mop})

\begin{lem}\label{lem:optional}If feature $f$ is an Optional subfeature  of feature $f^\prime$, the linear inequality for the Optional relationship is
  \begin{equation}
  |f^\prime| - |f| \ge 0
  \end{equation}
  \end{lem}
\begin{proof}
  $f \Rightarrow f^\prime$  is \emph{true} according to the optional relationship. Thus we can infer $\neg f  \lor f^\prime$  is \emph{true}. As a CNF, it can be converted to the inequality $(1-|f|)+|f^\prime| \ge 1$, that is $  |f^\prime| - |f| \ge 0$.
\end{proof}

\vspace{-3mm}
\begin{lem}\label{lem:mandatory}
If feature $f$ is a Mandatory subfeature  of feature $f^\prime$, the linear equality for the Mandatory relationship is
  \begin{equation}
  |f^\prime| - |f| = 0
  \end{equation}
\end{lem}
\begin{proof}
  $f \Leftrightarrow f^\prime$ is \emph{true} according to the mandatory relationship. Thus we infer the CNF $ (\neg f  \lor f^\prime) \land ( \neg  f^\prime \lor  f)$  is \emph{true}. Two inequalities are deduced:  $(1-|f|)+|f^\prime| \ge 1$ and $(1-|f^\prime| )+ |f|\ge 1$. By unifying them, we infer $|f^\prime| - |f| = 0$.
\end{proof}

\vspace{-2mm}
\begin{lem}If features $f_1$ ... $f_n$ are  the Or-subfeature of feature $f^\prime$, %the  equality for the Or relationship is
%  \begin{equation}
%  \prod\nolimits_{i=1}^{n}{(1- f_i)} + f^\prime  = 1
%  \end{equation}
%\noindent and
the linear inequalities for this relationship are
  \begin{equation}
  \forall  i \in  \{1,...,n\}, ~~ |f_i| -  |f^\prime|  \le 0  \label{formula:or1}
  \end{equation}
  \begin{equation}
  \sum\nolimits_{i=1}^{n}{|f_i|} - |f^\prime| \ge 0 \label{formula:or2}
  \end{equation}
\end{lem}
\begin{proof}
$\bigvee\nolimits_{i=1}^{n}({f_i})  \Leftrightarrow f^\prime $ is \emph{true} according to the Or relationship. Here, $\bigvee\nolimits_{i=1}^{n}({f_i})$ notates $f_1 \lor f_2 \lor ... \lor f_n $. Thus, the formulae $\bigvee\nolimits_{i=1}^{n}({f_i})  \Rightarrow f^\prime $ and $f^\prime \Rightarrow \bigvee\nolimits_{i=1}^{n}({f_i})$ need to be \emph{true}. For the first formula, we can represent it as the CNF and get the resulting formula $\bigwedge \nolimits_{i=1}^{n}({\neg f_i \lor f^\prime})$  to be \emph{true}. Note that for an indexed set of propositions $P = \{p_1,...,p_n\}$, $\bigwedge\nolimits_{i=1}^{n}(p_i)$ means that each proposition in $P$ needs to be \emph{true}.  Thus, we can get $n$ derived linear  inequalities that are listed in the inequality (\ref{formula:or1}). For the second formula $f^\prime \Rightarrow \bigvee\nolimits_{i=1}^{n}({f_i})$, we get $\neg f^\prime \lor \bigvee\nolimits_{i=1}^{n}({f_i})$ to be \emph{true}, that is $ \neg f^\prime \lor f_1 \lor ... \lor f_n $ to be true. So the inequality   (\ref{formula:or2}) can be deduced from the second formula.
\end{proof}



\begin{lem}If features $f_1$ ... $f_n$ are the Alternative subfeature of feature $f^\prime$, the linear inequalities for this relationship are
  \begin{equation}
  \forall  i \in  \{1,...,n\}, ~~ |f_i| -  |f^\prime|  \le 0
  \end{equation}
  \begin{equation}
  \sum\nolimits_{i=1}^{n}{|f_i|} - |f^\prime| \ge 0
  \end{equation}
  \vspace{-4mm}
  \begin{equation}
  \sum\nolimits_{i=1}^{n}{|f_i|}  \le 1 \label{formula:alter3}
  \end{equation}
  \end{lem}
\begin{proof}
Essentially, alternative subfeatures are a special type of \emph{or} features. In the subfeatures of an \emph{or} relationship, at least one subfeature needs to be selected. Whereas, only and exactly one needs to be selected in the alternative subfeatures. Instead of using inequalities  c(9)--c(11) in Table \ref{table:constrains}, the concise inequality (\ref{formula:alter3}) is  used to assure the exclusiveness of all subfeatures.
\end{proof}
Except the above 4 types of TCs, there are 3 types of CTCs. The \emph{requirement} of $f^\prime$ for $f$ can be formulated as $f \Rightarrow f^\prime$, and we can deduce $|f^\prime|-|f| \ge 0$ according to Lemma \ref{lem:optional}. Similarly, the \emph{iff} relationship between $f^\prime$ and $f$ is formulated as $f \Leftrightarrow f^\prime$, and we can deduce  $|f^\prime| - |f| = 0$ according to Lemma \ref{lem:mandatory}. Last, we have the last type of CTCs:

\begin{lem}If feature $f$ Excludes feature $f^\prime$, the linear equality for the Exclusion relationship is
  \begin{equation}
   |f^\prime| + |f| \le 1
  \end{equation}
  \end{lem}
\begin{proof}
$f \Rightarrow \neg f^\prime$ is \emph{true} according to the exclusion relationship. Thus we can infer $\neg f  \lor \neg f^\prime$  is \emph{true}. As a CNF, it can be converted to $(1-|f|)+(1-|f^\prime|) \ge 1$, that is $  |f^\prime| + |f| \le 1$.
\end{proof}

\subsection{The Integer Programming Model of Our Example}\label{sec:spl_ip_problem}
Let $\vec x$ be a solution, a binary variable vector $\vec x \in \{0,1\}^n$  where variable $x_i$ denotes $|f_i|$.
For each feature $f_i$, we denote its attributes $Used(f_i)$, $Defect(f_i)$ and $Cost(f_i)$ as coefficient $a_i$, $b_i$ and $c_i$, respectively. We denote the objective function for $obj_j$ as $\mathcal{F}_j(\vec x)$, $j \in \{1,...,k\}$. Subject to the linear inequalities converted from TCs and CTCs,
the example is formulated  as a multi-objective BIP (MOBIP) problem:

\begin{equation}\label{formula:mop}
%%\begin{array}{rrclcl}
%%\displaystyle \min_{x} & \multicolumn{3}{l}{c^T x} \\
%%\textrm{s.t.} & A x & \geq & b \\
%%%&\displaystyle \sum_{i=0}^{n} x_i & = & 1 \\
%%& x_i & \in & \{0,1\} & & \forall i \in \{1...n\} \\
\begin{array}{ll@{}r@{}r@{}l}
    \text{Min} & \mathcal{F}_2(\vec x)= \sum\nolimits_{i=1}^{n}(1-x_i) \\[\jot]
    \text{Min} & \mathcal{F}_3(\vec x)=\sum\nolimits_{i=1}^{n}(a_i\cdot x_i) \\[\jot]
    \text{Min} & \mathcal{F}_4(\vec x)=\sum\nolimits_{i=1}^{n}(b_i\cdot x_i) \\[\jot]
    \text{Min} & \mathcal{F}_5(\vec x)=\sum\nolimits_{i=1}^{n}(c_i\cdot x_i) \\[\jot]
    \text{s.t.} & \text{the inequalities for $conj(M)$ hold}\\[\jot]

%    &  x_1 - x_3  \ge 0   \\
%    &  x_1 - x_4 \ge 0      \\
%    &  x_1 - x_5 \ge 0      \\
%    &  x_1 - x_6 \ge 0      \\
%    &  x_1 - x_7 \ge 0      \\
%    &  x_2 - x_8 \ge 0      \\
%    &  x_2 - x_9 \ge 0      \\
%    &  x_2 - x_{10} \ge 0      \\
%    &  -x_2 + x_8 + x_9 + x_{10}\ge 0      \\
%    &  -x_8 - x_9 - x_{10}\ge -1      \\
%    &  x_6 - x_{11} \ge 0      \\
%    &  x_6 - x_{12} \ge 0      \\
%    &  -x_6 + x_{11} + x_{12} \ge 0      \\
%    &  x_7 - x_{11} \ge 0      \\
%    &  x_7 - x_{12} \ge 0      \\
%    &  -x_7 + x_{11} + x_{12} \ge 0      \\
  \end{array}
%%\end{array}
\end{equation}
%s.t. the inequalities for c(1)--c(13) in Table \ref{table:constrains} hold.\\

\section{$\epsilon$-constraint Method and CWMOIP}\label{sec:naiveSolutions}

When the problem size is relatively small, the $\epsilon$-constraint method and CWMOIP are two feasible methods to find the complete non-dominant solutions (a.k.a., true Pareto front).

\begin{algorithm}[t]
                  % enter the algorithm environment
\caption{Function \small{$EpsilonCont()$} for feature selection}\label{alg:naiveLP}
\small
    \KwIn{$M$: the feature model of the given system} %$maxIter$: the maximum number of iterations of the generation process
    %\KwIn{$userInfor$: the contextual information of user device} %$maxIter$: the maximum number of iterations of the generation process
    \KwOut{$solutions$: a non-dominated solution set for feature selection}
    %\KwOut{$returnedSol$: a solution returned to guide malware generation }
    $E~\leftarrow~\emptyset$\;\label{algo:lp:i1}
    $f^{TUB}_2 = getObjTheoBound(M,\mathcal{F}_2), f^{TLB}_2 = 0$\;\label{algo:lp:i2}
    $f^{TUB}_3 = getObjTheoBound(M,\mathcal{F}_3), f^{TLB}_3 = 0 $\;\label{algo:lp:i3}
    $f^{TUB}_4 = getObjTheoBound(M,\mathcal{F}_4), f^{TLB}_4 = 0 $\;\label{algo:lp:i4}

     \For{$ p=f^{TLB}_2;   p \le f^{TUB}_2; p = p \small{+} 1$}{\label{algo:lp:f1}
        \For{$ q= f^{TLB}_3;  q \le f^{TUB}_3; q = q\small{+} 1$}{\label{algo:lp:f2}
          \For{$ t=f^{TLB}_4;  t \le f^{TUB}_4; t = t\small{+} 1$}{\label{algo:lp:f3}
          %$conj(\mathcal{F}_2) = convert(\mathcal{F}_2,p) $\; \label{algo:lp:lp1}
          %$conj(\mathcal{F}_3) = convert(\mathcal{F}_3,q) $\; \label{algo:lp:lp2}
          %$conj(\mathcal{F}_4) = convert(\mathcal{F}_4,t) $\; \label{algo:lp:lp3}
          $allCons= conj(M) \cup \{\mathcal{F}_2\le p\} \cup \{\mathcal{F}_3\le q\}  \cup \{\mathcal{F}_4 \le t\} $\;\label{algo:lp:lp4}
          $ME = bintprog(allCons,\mathcal{F}_5) $\; \label{algo:lp:lp5}
          $E = E \cup ME$\;  \label{algo:lp:lp6}
          }
        }
     }
    % $returnedSol = solutions.First()$\;
   %  \For{$sol  \in  nondominatedSol$}{  \label{algo:lp:f3}
%        \If {$  aggregatedObj(sol) < aggregatedObj(returnedSol)$} {               \label{algo:lp:if1}
%       %  \If {$ satisfyExtraConst(sol) == true $} {               \label{algo:lp:if2}
%          $returnedSol = sol$\; \label{algo:lp:if3}
%       %   }
%        }
%     }

    \KwRet $E$;\label{algo:lp:rt}
      %  $allMal\leftarrow allMal \cup newGeneration$\;\label{algo:cmb:23}

\end{algorithm}

\subsection{$\epsilon$-constraint Method}
The idea is to make $k-1$ objectives as the range constraints and use the $k$-th one as the objective function in BIP \cite{e-constraint}. As \emph{obj1} is correctness, in BIP, all the constraints are satisfied, \emph{obj1} is always 0 in theory and thus $\mathcal{F}_1(\vec x)$ is not needed. So $\mathcal{F}_2(\vec x)$, $\mathcal{F}_3(\vec x)$, $\mathcal{F}_4(\vec x)$ can be converted to range constraints. Each range constraint's upper bound will be iterated from 0 to the upper bound of the corresponding objective, by step size 1.

The detailed procedures are shown in Algorithm \ref{alg:naiveLP}. Note that $getObjTheoBound(M,\mathcal{F}_2)$ at line \ref{algo:lp:i2} finds the theoretic upper bound $f^{TUB}_2$ for $\mathcal{F}_2$ --- for \emph{JCS}, it is 12 when all $x_i=0$  and $conj(M)$ is not considered. Similarly, $f^{TUB}_3$ and $f^{TUB}_4$ are $\sum\nolimits_{i=1}^{n}a_i$ and $\sum\nolimits_{i=1}^{n}b_i$ respectively,  when the size of feature set $n = \mathit{\vert Fea(JCS) \vert} =12$ for \emph{JCS}. At line \ref{algo:lp:lp4},  $allCons$ is the union of the original inequalities of the formula (\ref{formula:mop}) and three new inequalities converted from other constrained objectives in formula  (\ref{formula:singLP}). At line \ref{algo:lp:lp5}, $bintprog(allCons,\mathcal{F}_5)$ calls the BIP function for objective $\mathcal{F}_5$ such that $allCons$ are satisfied.

\vspace{-3mm}
\begin{equation}\label{formula:singLP}
%%\begin{array}{rrclcl}
%%\displaystyle \min_{x} & \multicolumn{3}{l}{c^T x} \\
%%\textrm{s.t.} & A x & \geq & b \\
%%%&\displaystyle \sum_{i=0}^{n} x_i & = & 1 \\
%%& x_i & \in & \{0,1\} & & \forall i \in \{1...n\} \\
\begin{array}{ll@{}r@{}r@{}l}
     \text{Min} & \mathcal{F}_5(\vec x) =\sum\nolimits_{i=1}^{n}(c_i\cdot x_i) \\[\jot]
     \text{s.t.} & \sum\nolimits_{i=1}^{n}(1-x_i) \le p \\[\jot]
      & \sum\nolimits_{i=1}^{n}(a_i\cdot x_i) \le q \\[\jot]
      & \sum\nolimits_{i=1}^{n}(b_i\cdot x_i) \le t\\[\jot]
       & \text{the inequalities for $conj(M)$ hold}\\[\jot]
%    &  x_1 - x_3  \ge 0   \\
%    &  x_1 - x_4 \ge 0      \\
%    &  x_1 - x_5 \ge 0      \\
%    &  x_1 - x_6 \ge 0      \\
%    &  x_1 - x_7 \ge 0      \\
%    &  x_2 - x_8 \ge 0      \\
%    &  x_2 - x_9 \ge 0      \\
%    &  x_2 - x_{10} \ge 0      \\
%    &  -x_2 + x_8 + x_9 + x_{10}\ge 0      \\
%    &  -x_8 - x_9 - x_{10}\ge -1      \\
%    &  x_6 - x_{11} \ge 0      \\
%    &  x_6 - x_{12} \ge 0      \\
%    &  -x_6 + x_{11} + x_{12} \ge 0      \\
%    &  x_7 - x_{11} \ge 0      \\
%    &  x_7 - x_{12} \ge 0      \\
%    &  -x_7 + x_{11} + x_{12} \ge 0      \\
  \end{array}
\vspace{-3mm}
%%\end{array}
\end{equation}

Algorithm \ref{alg:naiveLP} is of the time complexity of $O(n^3)$, if considering BIP solving function $bintprog()$ takes constant time --- a time limit is set in its practical usage. Solving the MOBIP problem in formula (\ref{formula:mop}) is reduced to solving the BIP problem in formula (\ref{formula:singLP}) by many times --- precisely, it is a number of $(n+1)  ({\sum\nolimits_{i=1}^{n}a_i}+1)({\sum\nolimits_{i=1}^{n}b_i}+1)$ times.



\subsection{CWMOIP}
CWMOIP, proposed by  \"{O}zlen \emph{et al.}~\cite{DBLP:journals/eor/OzlenA09}, is an objective-reduction technique for MOIP. CWMOIP is for generating \emph{all} non-dominant solutions.
It improves the $\epsilon$-constraint method by two steps: (1) for each objective, the lower bound is not 0 and the upper bound is not the sum of feature attributes revelent to that objective. To be precise, BIP is applied to get the true lower and upper bounds for each objective (i.e., $\mathcal{F}_2$ to $\mathcal{F}_5$) separately, subject to the  conjunction of constraints $conj(M)$. (2) objective-reduction is implemented via the constraint weight method, to avoid generating dominant solutions. The $k$-objective problem is first reduced to that of $k-1$, then $k-2$, iteratively, until the last objective.

\noindent\textbf{Example for (1).} In the precise calculation of the upper and lower bounds, for the example of \emph{JCS} with 12 features, the true bounds $f^{LB}_2$ and $f^{UB}_2$ for $\mathcal{F}_2(\vec x)$ are 2 (a maximum of 10 selected  features) and 9 (a minimum of 3 selected) respectively, not 0 and 12 in the $\epsilon$-constraint method. Hence, in $\epsilon$-constraint method 13 (12-0+1) times of iteration is needed for the outermost loop at line~\ref{algo:lp:f1} in Algorithm \ref{alg:naiveLP}, while  only 8 (9-2+1) times is needed for CWMOIP.


\begin{algorithm}[t]                  % enter the algorithm environment
\caption{Function \small{\emph{CWMOIP()}} for $k$-objective IP}\label{alg:cwmoip}
\small
    %\KwIn{$M$: the feature model of the given system} %$maxIter$: the maximum number of iterations of the generation process
    \KwIn{$k$: the number of objectives, $l_k$: constrained value for the weighted $k$-th \emph{obj}, $X$: the set of linear constraints}
    %\KwIn{$userInfor$: the contextual information of user device} %$maxIter$: the maximum number of iterations of the generation process
    \KwOut{$E$: the set of non-dominant solutions}
    %\KwOut{$returnedSol$: a solution returned to guide malware generation }
    $E~\leftarrow~\emptyset$\;\label{algo:cwmoip:i1}
    $f^{UB}_2,f^{LB}_2  = getObjTrueBound(X,f_2)$\;\label{algo:cwmoip:i2}
    $ ...$//get true bounds for other objective $3$ to $k-1$\;\label{algo:cwmoip:i3}
      $f^{UB}_k,f^{LB}_k  = getObjTrueBound(X,f_k)$\;\label{algo:cwmoip:i4}
      $w_k =\frac{1}{(f^{UB}_2-f^{LB}_2+1)(f^{UB}_3-f^{LB}_3+1)...(f^{UB}_k-f^{LB}_k+1)}$ \;\label{algo:cwmoip:i5}
    %   $l_k = f^{UB}_k$\;\label{algo:cwmoip:65}
      \If{$ k =1$}{\label{algo:cwmoip:i6}
           $E =E \cup bintprog(X,f_k)$ \;\label{algo:cwmoip:i6:if1}
       }
     \While{$true$}{\label{algo:cwmoip:f1}
         $ f_1 = addObjFuncSuffix(f_1,w_k \cdot f_k ) $\;\label{algo:cwmoip:f2}
         $ X^\prime = X \cup \{f_k \le l_k\}$\;\label{algo:cwmoip:f3}
         $ ME= CWMOIP(k-1,l_k,X^\prime) $ \;\label{algo:cwmoip:f4}
         \If{$ME = Null \empty$ }{\label{algo:cwmoip:f5}
             $break$\;\label{algo:cwmoip:if2}
         }
          $   E = E \cup ME$\;\label{algo:cwmoip:f6}
         $l_k = Max(f_k(\vec x),\vec x \in ME) -1$\;\label{algo:cwmoip:f7}

     }
    % $returnedSol = solutions.First()$\;
   %  \For{$sol  \in  nondominatedSol$}{  \label{algo:lp:f3}
%        \If {$  aggregatedObj(sol) < aggregatedObj(returnedSol)$} {               \label{algo:lp:if1}
%       %  \If {$ satisfyExtraConst(sol) == true $} {               \label{algo:lp:if2}
%          $returnedSol = sol$\; \label{algo:lp:if3}
%       %   }
%        }
%     }

    \KwRet $E$;\label{algo:cwmoip:rt}
      %  $allMal\leftarrow allMal \cup newGeneration$\;\label{algo:cmb:23}

\end{algorithm}



\begin{figure}[t]
\vspace{-5.5mm}
\centering
%\epsfig{file=image/fm.bmp, width=8.5cm}
\includegraphics[width=7.8cm]{image/CWMOIP.png}
\vspace{-3.5mm}
\caption{An example of applying CWMOIP for a $bi$-objective problem}
\label{fig:cwmoip_exam}
\end{figure}


\noindent\textbf{Example for (2).}  Fig.~\ref{fig:cwmoip_exam} shows an example for objective-reduction. Solving the $bi$-objective problem in formula (15)  is reduced to solving the 1-objective problem in formula (16) by \rev{$l_2$} times. It is named "constraint weighted" because of the weight $w_2$ and the constraint $ f_2(x) \le l_2$. %on the original objective $ f_2(x)$.
 Variable $l_2$ iterates from the lower bound $f^{LB}_2$ to the upper bound $f^{UB}_2$ of $ f_2(x)$. %Here, $f^{LB}_2$ (or $f^{UB}_2$) denotes the lower (or upper) bound  of $ f_2(x)$.}
%Interested readers can refer to \cite{DBLP:journals/eor/OzlenA09} for the details of the proof.



In Algorithm \ref{alg:cwmoip}, we show the general steps of finding all non-dominant solutions for any given $k$-objective BIP problem~\cite{DBLP:journals/eor/OzlenA09} (not limited to the model of our problem). The initial invocation of Algorithm \ref{alg:cwmoip} is calling $CWMOIP(k,f^{UB}_k,X)$, and then $CWMOIP(k-1,l_k,X)$ at line \ref{algo:cwmoip:f4}, recursively, until calling $CWMOIP(1,l_2,X)$. %We explain the important steps in Algorithm 2.
Initially, lines \ref{algo:cwmoip:i2} to \ref{algo:cwmoip:i4} calculate the true upper and lower bounds of the $2$-nd objective to the $k$-th, subject to the constraint set $X$ (In practice, this can be done once and results are cached for reuse). Then $w_k$ is calculated for the $k$-th objective at line \ref{algo:cwmoip:i5}. In the loop at line \ref{algo:cwmoip:f1}, the $k$-objective problem is reduced to a new \emph{(k-1)}-objective problem (line \ref{algo:cwmoip:f4}), which has the new suffix $w_k\cdot f_k(\vec x)$ for the objective function (line \ref{algo:cwmoip:f2}) and the new constraint $f_k(\vec x) \le l_k$ for the constraint set $X$ (line \ref{algo:cwmoip:f3}). If no results are found (lines \ref{algo:cwmoip:f5}-\ref{algo:cwmoip:if2}), the recursion process stops. If found, the constraint $l_k$ is tightened to the value just smaller than the largest value of  $f_k(\vec x)$ for $\vec x \in ME$. Last, BIP solving function is called when only one objective ($k=1$) is left at line \ref{algo:cwmoip:i6} to \ref{algo:cwmoip:i6:if1}.

According to \cite{DBLP:journals/eor/OzlenA09}, the maximum number of recursion is $\frac{|E|(|E|+1)...(|E|+k-2)}{2\cdot 3 \cdot ... \cdot(k-1)}$.
Note that in our example, $\mathcal{F}_1(\vec x)$ is not needed as it is always   $0$ for BIP. Thus, it is a 4-objective ($\mathcal{F}_2(\vec x)$ to $\mathcal{F}_5(\vec x)$) BIP problem. $CWMOIP()$ will reduce it to constrain-weighted 3-objective ($\mathcal{F}_2(\vec x)$ to $\mathcal{F}_4(\vec x)$); iteratively, until to a constrain-weighted 1-objective ($\mathcal{F}_2(\vec x)$) problem.
%\begin{equation}\label{formula:singLP}
%%%\begin{array}{rrclcl}
%%%\displaystyle \min_{x} & \multicolumn{3}{l}{c^T x} \\
%%%\textrm{s.t.} & A x & \geq & b \\
%%%%&\displaystyle \sum_{i=0}^{n} x_i & = & 1 \\
%%%& x_i & \in & \{0,1\} & & \forall i \in \{1...n\} \\
%\begin{array}{ll@{}r@{}r@{}l}
%     \text{Min} & f_1(x) \\[\jot]
%     \text{Min} & f_2(x) \\[\jot]
%     \text{s.t.}& x \in X\\[\jot]
%  \end{array}
%\vspace{-3mm}
%%%\end{array}
%\end{equation}
%
%%\subsection{Preliminary Evaluation Results}
%\begin{equation}\label{formula:singLP}
%%%\begin{array}{rrclcl}
%%%\displaystyle \min_{x} & \multicolumn{3}{l}{c^T x} \\
%%%\textrm{s.t.} & A x & \geq & b \\
%%%%&\displaystyle \sum_{i=0}^{n} x_i & = & 1 \\
%%%& x_i & \in & \{0,1\} & & \forall i \in \{1...n\} \\
%\begin{array}{ll@{}r@{}r@{}l}
%     \text{Min} & f_1(x) + w_2 \cdot f_2(x)\\[\jot]
%     \text{s.t.} &  x \in X \\[\jot]
%     \text{s.t.}& f_2(x) \le l_2\\[\jot]
%     \text{s.t.}& w_2 = 1 / (f^{UB}_2-f^{LB}_2+1) \\[\jot]
%  \end{array}
%\vspace{-3mm}
%%%\end{array}
%\end{equation}
%

\section{Representation Generation Approach --- \ourSol}\label{sec:solution}
%In literature, there are three main approaches for solving multi-objective optimization problems: exact methods, meta-heuristic methods \cite{deb2001} and approximation methods \cite{ruzika2005approximation}. Exact methods aim to generate the entire Pareto front. {\color{red} i suggest to remove this paragraph, it seems to be not quite relevant here}
\vspace{-1mm}
The optimal feature selection problem is essentially a combinatorial problem. %Due to the non-convex nature of such problems, it is typically infeasible to obtain a closed-form representation of the Pareto front via CWMOIP \rev{[??]}. Further,
The number of non-dominant solutions grows exponentially along the number of features and objectives. Given its NP-hardness, it is neither practical nor necessary to find the entire true Pareto front \cite{koksalan2009multiobjective}. %as $\epsilon$-constraint approach and CWMOIP do.
Instead, obtaining a set of solutions that evenly distributed over the true Pareto front is representative, pragmatic and computationally referrable.

%\subsection{\ourSol~Method}\label{sec:solution:nc}
The normal constraint (NC) method \cite{normalCons} is an effective method in the literature that guarantees generating evenly distributed solutions over the entire Pareto front. The main idea of NC method is to use the \emph{utopia plane} to approximate the true Pareto front such that a set of evenly distributed reference points on \emph{utopia plane} would result to a set of evenly distributed solutions on Pareto front, after the projection along the normal vector of \emph{utopia plane}. Here, \emph{utopia plane} refers to the hyper plane determined by all the anchor points, each of which individually optimizes a single objective (e.g. $y_{1}^*$ for $y_1$-axis and $y_{2}^*$ for $y_2$-axis in Fig~\ref{fig:uhre}.a). However, the original NC method was developed on continuous solution space. In the case of IP, it can generate dominated solutions (see Fig~\ref{fig:uhre}.a) where point A is the solution generated using NC, but there exists a point B outside the shaded feasible region, dominating A. Therefore, we propose to combine the idea of \emph{utopia plane} and the $\epsilon$-constraint method such that all solutions are guaranteed to be non-dominant ones (see Fig~\ref{fig:uhre}.c).

\begin{figure}[t]
\vspace{-1mm}
\centering
%\epsfig{file=image/fm.bmp, width=8.5cm}
\includegraphics[width=8cm]{image/uhre.png}
\vspace{-3mm}
\caption{Two dimensional illustration of the proposed \ourSol~method}
\label{fig:uhre}
\end{figure}

Our proposed method consists of four main steps: 1) determine the utopia plane; 2) generate uniformly distributed reference points on the utopia plane; 3) for each reference point, use its $(k-1)$ coordinates to constrain the $(k-1)$ objectives; 4) optimize the $k$-th objective within the reduced solution space. Repeat steps 2) to 4) to achieve the representative Pareto front. In step 2), we resort to the hit-and-run (H\&R) method \cite{DBLP:journals/ior/Smith84} to sample the reference points.
%\rev{please insert this paper to reference list: Smith R L. Efficient Monte Carlo Procedures for Generating Points Uniformly Distributed Over Bounded Regions[J]. Operations Research, 1984, 32(6):1296-1308.}
Its principle is straightforward: let $p_{t}$ denote the current point then $p_{t+1} = p_{t} + \lambda*d$ is the next point, where $d$ is a random direction vector and $\lambda$ is the random length of the jump. As a random-walk algorithm, H\&R is proven to generate uniformly distributed points inside any polyhedron after a sufficient number of runs \cite{DBLP:journals/ior/Smith84}.

Note that an obvious alternative is to equally divide the Pareto front along each objective direction and then traverse each grid. However, it might not evenly cut the Pareto front and the computational efforts grow exponentially along the number of objectives.

\begin{algorithm}[t]                  % enter the algorithm environment
\caption{Function \small{\emph{SolRep()}} for $k$-objective IP}\label{alg:nchr}
\small
    %\KwIn{$M$: the feature model of the given system} %$maxIter$: the maximum number of iterations of the generation process
    \KwIn{$k$: the number of objectives, $N$: number of reference points, $X$: the set of linear constraints, $f_{i}$ coefficient vector of the $i$-th objective function}
    %\KwIn{$userInfor$: the contextual information of user device} %$maxIter$: the maximum number of iterations of the generation process
    \KwOut{$E$: the set of representative non-dominant solutions}
    %\KwOut{$returnedSol$: a solution returned to guide malware generation }
    \For{$ i = 1; i < k; i=i+1 $}{ \label{algo:nchr:i1}
    $y_{anch,i}= bintprog(X,f_{i})$\; \label{algo:nchr:i2}
    }
    $ ...$//determine the vertexes of utopia plane\; \label{algo:nchr:i3}
    $p_{0} = randPoint(y_{anch})$\; \label{algo:nchr:i4}
    $ ...$//generate a initial reference point $p_0$ on utopia plane\; \label{algo:nchr:i5}
    $P~\leftarrow~\emptyset, P = P \cup \{p_{0}\}$\; \label{algo:nchr:i6}
    \For{$ i = 1; i < N; i=i+1 $}{ \label{algo:nchr:i7}
    $d = randDirect(y_{anch})$\; \label{algo:nchr:i8}
    $\lambda_{range} = linprog(X,d,p_{0})$\; \label{algo:nchr:i9}
    $\lambda = unifrnd(\lambda_{range})$\; \label{algo:nchr:i10}
    $p_{0} = p_{0}+\lambda * d $\; \label{algo:nchr:i11}
    $P = P \cup \{p_{0}\}$\; \label{algo:nchr:i12}
    }
    $ ...$//find the other N-1 reference points\; \label{algo:nchr:i13}
    $E~\leftarrow~\emptyset$\; \label{algo:nchr:i14}
    \For{$ i = 1; i \le N; i=i+1 $}{ \label{algo:nchr:i15}
    $E = E \cup bintprog(X,f_{k},P_{i})$ \label{algo:nchr:i16}
    }
    \KwRet $E$;\label{algo:cwmoip:rt}
      %  $allMal\leftarrow allMal \cup newGeneration$\;\label{algo:cmb:23}

\end{algorithm}

The proposed \ourSol~method is presented in Algorithm \ref{alg:nchr}, which has the time complexity of $O(n)$ (consider $bintprog(X,f_{i})$ has $O(1)$, as a time limit is set for IP solving). The function $bintprog(X,f_{i})$ optimizes the $i$-th objective and returns one anchor point, $randPoint(y_{anch})$ generates a random initial reference point on utopia plane, $randDirect(y_{anch})$ returns a random direction vector, $linprog(X,d,p_{0})$ returns the upper and lower bounds of the jumping length $\lambda$, $unifrnd(\lambda_{range})$ returns one random length within the bounds, $bintprog(X,f_{k},P_{i})$ optimizes the $k$-th objective with the constraints incurred by the reference point $P_{i}$.
%As mentioned in Introduction Section, the drawbacks of meta-heuristic methods are: 1) non-guarantee of achieving the true Pareto optimal solutions; 2) the diversity of the obtained solutions over the entire solution space might not be guaranteed also.

%The normal boundary intersection (NBI) method \cite{das1998normal}. It generates non-Pareto and locally Pareto solutions \cite{messac2004normal}.
%The normal constraint (NC) method \cite{messac2004normal} guarantees the evenly distributed solutions over the entire Pareto front. In this method, the MOO problem is converted into one single objective optimization problem which is solved iteratively subject to a properly designed set of constraints. Different from the normal boundary intersection method, which requires the solutions being on the normal line.

%A comparative study by Messac et al. \cite{messac2003normalized} shows that NC performs favorably contrasting against NBI and physical programming. It is more computationally stable and is less likely trap to non-Pareto or locally Pareto solutions.


\input{section/results}
%\subsection{Discussion}\label{sec:application:discuss}


\noindent\textbf{Threats to Validity.}
One threat to validity is about randomly generated values for feature attributes (i.e.,  Cost,  Defects,  and  Used  Before). %Due to the difficulty in getting feature attributes  associated with  real-world  products,
To  mitigate  the  effect  of  randomness, we  generate  10  sets  of  attributes for each model. Due to the page limit, we just report 2 or 4 representative attribute sets in the evaluation. According to \cite{DBLP:journals/asc/XueZT0CC016} and our observation, impact of attribute values is minor to the results ---  if on two sets of attributes a method is better; then in general (like on ten sets) this method is better. %Besides, for each set, IBED or IBEA is repeatedly executed for 30 times, and  report  the  medium  values  of the  metrics.
The second threat is about the systems chosen in evaluation. In future, the \emph{EC2}  feature  model \cite{DBLP:journals/fgcs/Garcia-GalanTRC16} and  the  \emph{Drupal} model  \cite{Sanchez2015} need to be included in the evaluation. The last threat is about the parameters of the EAs used as baseline tools. We used the best parameter setting of IBED (and also IBEA) reported by \cite{DBLP:journals/asc/XueZT0CC016}.

\vspace{-2mm}
%\noindent\textbf{Generality of \ourSol.} \ourSol~is proven to be effective and scalable in solving the optimal feature selection in SBSE. We are eager to try out \ourSol~for other MOO problem, as long as the constraints and multiple objectives are linear. As \ourSol~is designed as a general method for MOIP with linear constraints, some existing benchmarks [??] can be used to evaluate \ourSol. We plan to submit \ourSol~to OR community for review.
%\subsection{Heuristic, Analytic or Statistic?}\label{sec:discuss:how}

\section{Related Work}\label{sec:related}

\noindent\textbf{The Optimal Feature Selection Problem.}
White \emph{et al.}~\cite{DBLP:journals/jss/WhiteDS09} first modeled the feature selection problem as a Multidimensional Multi-Choice Knapsack Problem (MMKP), and applied Filtered Cartesian Flattening (FCF) to derive an optimal feature configuration subject to resource constraints.
%The existing works \cite{DBLP:journals/jss/GuoWWLW11,DBLP:conf/icse/SayyadMA13,conf/cmsbse/SayyadMA13,DBLP:dblp_conf/kbse/SayyadIMA13, DBLP:conf/issta/TanXCSLD15} have adopted evolutionary algorithms (EAs) for feature selection with resource constraints and product generation based on the value of user preferences, respectively.
Guo \emph{et al.}  \cite{DBLP:journals/jss/GuoWWLW11} first proposed a genetic algorithm (GA) approach to tackle this problem. In \cite{DBLP:journals/jss/GuoWWLW11}, a repair operator is used to fix each candidate solution, and make it comply with the feature model during evolution. %This approach might be non-terminating, and furthermore, it does not take advantage of the automatic correction  that brought by the GA.
One limitation of \cite{DBLP:journals/jss/GuoWWLW11} lies in aggregating all objectives into a single fitness function with different weights. %This only gives users a solution specific to the weights used in the objective formula.

To address the objective aggregation issue, Sayyad \emph{et al.}~\cite{DBLP:conf/icse/SayyadMA13,conf/cmsbse/SayyadMA13} first proposed to apply various MOEAs, and a range of optimal solutions (a.k.a., a Pareto front) is returned to the user as a result. As reported, IBEA \cite{DBLP:conf/ppsn/ZitzlerK04} yields the best results among the seven tested EAs in terms of time, correctness and satisfaction to user preferences. In \cite{DBLP:dblp_conf/kbse/SayyadIMA13}, they further used static method to prune features before execution of IBEA for reducing search space. They also introduced a ``seeding method\rq\rq{} by pre-computing a correct solution, which was later implanted the initial population of IBEA. Along this line, Tan \emph{et al.} \cite{DBLP:conf/issta/TanXCSLD15} improved these previous studies by using a novel feedback-directed mechanism to existing EAs. %In their approach, the feature model is first preprocessed based on SAT solving to remove the prunable features, before the execution of an EA. %We have shown that we always prune more features compared to the pruning method in~\cite{DBLP:dblp_conf/kbse/SayyadIMA13}.
%During evolution, the violated constraints would be analyzed and analyzed results are used as feedback to guide evolutionary operators (i.e., crossover and mutation) for producing correct offsprings. %for the next round. This feedback-directed mechanism is mainly used to improve the correctness of offsprings.
Similarly, to improve the correctness, Hierons \emph{et al.} \cite{DBLP:journals/tosem/HieronsLLSZ16} proposed the $1+n$ approach that prioritizes the number of failed constraints and considers the correctness objective first.
%Our evaluation has shown that our method produces more promising offsprings (that have fewer violated constraints), which has led to faster convergence and resulted in more valid solutions in a significantly shorter amount of time.
Recently, IBEA (including its variants) is integrated with other techniques for achieving better results.  Henard \emph{et al.}~\cite{DBLP:conf/icse/HenardPHT15}
integrated IBEA with constraint solving. They permuted different SAT parameters to maximize the diversity of SAT solutions in a cheap way by calling SAT solver hundreds of times. %The merit is that the fixing of incorrect offspring of IBEA population via SAT solutions is done during evolution.
Xue \emph{et al.}~\cite{DBLP:journals/asc/XueZT0CC016} integrated IBEA with differential evolution (DE) for achieving both correctness and diversity of solutions. % Their method, named IBED, is a dual-population EA, where two populations are evolved with two different types of EA operators, i.e., IBEA operators and DE operators.

\vspace{0.5mm}
\noindent\textbf{MOO and SBSE.}
Apart from the above problem, MEOAs have exhibited the effectiveness in solving some earlier SBSE problems~\cite{DBLP:journals/csur/HarmanMZ12}. For example,  multi-objective planning for software project overtime is the SBSE problem solved by a variant of NSGAII \cite{DBLP:conf/icse/FerrucciHRS13}. Besides, the problem of bi-objective software effort estimation is also addressed by NSGAII~\cite{DBLP:conf/icse/SarroPH16}.
 In software refactoring, multi-objective refactoring step recommendation via NSGAII helps to ensure the semantic coherence of refactored program~\cite{DBLP:journals/ese/MkaouerKCHD17}. Last, a branch of MOO studies is on software test suite selection, minimization and prioritization~\cite{DBLP:conf/issta/YooH07}\cite{DBLP:journals/tse/MarchettoIASS16}.
NSGAII, which is used in aforementioned studies, is suitable  for  more  spread  out solutions  and  absolute domination. Hence, NSGAII works well if the solution space is not highly-constrained and diversity can help find more solutions. However, for the optimal feature selection, the preference of diversity (using NSGAII) may bring more incorrect solutions (since correct solutions may be concentrated in certain areas of  the solution space). To assure this, IBEA is advocated~\cite{DBLP:conf/icse/SayyadMA13}\cite{conf/cmsbse/SayyadMA13}. %IBEA is also combined with DE to achieve both correctness and diversity~\cite{DBLP:journals/asc/XueZT0CC016}.

%In \cite{DBLP:journals/infsof/HarmanJ01}, Harman \emph{et al.} proposed the term Search-Based Software Engineering (SBSE), and reported that the surveyed and proposed optimization techniques for SE problems by 2001 were all single-objective based. Seeing the potential of using multi-objective optimization, Harman \cite{DBLP:conf/icse/Harman07} discussed about the possible usage of the meta-heuristic search techniques such as: simulated annealing and genetic algorithm. Harman considered it insensible combination of  multiple metrics into an aggregate fitness in the way of assigning coefficients, and further suggested to use Pareto optimality rather than aggregate fitness.

\vspace{0.5mm}
\noindent\textbf{OR and SBSE.} IP has been used for the instantiation of products --- the valid product feature selection problem~\cite{DBLP:conf/splc/Broek10}. In~\cite{DBLP:conf/splc/Broek10}, only one single objective is considered, not addressing MOO via IP. Earlier than this study, requirement interdependencies were resolved via IP\cite{DBLP:journals/re/Carlshamre02}.
Then, flexible release planning was solved by IP~\cite{DBLP:conf/caise/AkkerBDV05}.
Subsequently, requirements selection and scheduling for the release planning were integrated and optimized by IP to cater for budgetary constraints  \cite{DBLP:conf/refsq/LiABD07}. Further,  IP was combined with computational intelligence and human negotiation to address conflicting objectives \cite{DBLP:journals/software/RuheS05}.
Recently, IP is not widely used due to its scalability issues and strict limitation on the application. Meanwhile, due to the emergence of MOO problems in SBSE, MOEAs  become the default method.

 %In 2009, Harman \emph{et al.}~\cite{tr:sesw} reported that MEOAs had been deployed to attain multi-objective optimization (especially two objectives), and the mainly used techniques were NSGA-II and SPEA2. By them, few studies had examined the performance and suitability of the commonly used MEOAs on more than three-objective optimization. In 2010, Bowman \emph{et al.}~\cite{DBLP:journals/tse/BowmanBL10} solved class responsibility assignment problem by applying five-objective optimization techniques on UML model analysis. They reported that SPEA2 can generally attain better results than RMHC1 and RMHC2. Recently, Sayyad \emph{et al.}~\cite{conf/raise/SabouriK11} found that among the existing application of MEOAs, the population based algorithm like NSGA-II was widely chosen for optimization of two or three objectives. Meanwhile, the comprehensive discussion and comparison of the suitability of different MEOAs are absent.
\vspace{-2mm}

%\input{relatedOld}
%\vspace{-1mm}
\section{Conclusion}\label{sec:conclusion}
In this paper, we formulate the optimal feature selection problem in SBSE as a MOIP model. Different from all previous studies that adopt MOEAs, we seek to apply MOIP methods. We try the \naiveSol~and CWMOIP, and prove the completeness of their solutions on small systems. However, they are not scalable. To address this, we propose an innovative MOIP method --- \ourSol. Evaluation shows that \ourSol~is scalable and finds significantly more non-dominant solutions than IBED in most cases, using similar or less time. In future, we will apply \ourSol~to more suitable SBSE problems.

\newpage

\bibliographystyle{ACM-Reference-Format}
\bibliography{section/reference}

\end{document}
